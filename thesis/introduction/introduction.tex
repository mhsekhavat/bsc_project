\فصل{مقدمه}
\برچسب{chap:intro}

فرآیند \واژه{تست.ن.ا.} از اولین پیدایش تا کنون با رویکرد‌ها و کاربردهای
متنوعی اجرا شده‌است. در این پایان‌نامه بر رویکرد
بهره‌گیری از توصیف نرم‌افزار به زبان مشتری در \واژه{تست} تمرکز شده و
معماری‌ جدیدی را برای \واژه{تست} با \واژه{تکنیک} \واژه{بی.دی.دی.}، که بر اساس این
رویکرد ارائه شده، پیشنهاد می‌کند.

گرچه ایده‌ی مطرح شده در معماری پیشنهادی منحصر به  \واژه{تکنیک}
\واژه{بی.دی.دی.} نیست و با سایر  \واژه{تکنیک}‌ها و \واژه{متدولوژی}‌ها نیز قابل انطباق است،
حوزه‌ی بررسی‌این پایان‌نامه به \واژه{تکنیک} \واژه{بی.دی.دی.} محدود شده
است.

در بخش \رجوع{sec:intro:history} تاریخچه‌ای از رویکردهای موجود نسبت به
\واژه{تست.ن.ا.} ارائه می‌شود. در بخش ‌های \رجوع{sec:intro:tdd} و
\رجوع{sec:intro:bdd} به ترتیب ادبیات مورد نیاز و جزییات \واژه{تکنیک}
\واژه{بی.دی.دی.} شرح داده می‌شوند.


\قسمت{تاریخچه}\برچسب{sec:intro:history}

روند تکامل رویکرد رایج در \واژه{تست.ن.ا.}، از سال ۱۹۵۷ تا سال ۲۰۰۰ به پنج دوره‌ی \نام{رفع‌اشکال-محور}{Debugging-oriented}،
\نام{اثبات-محور}{Demonstration-oriented}،
\نام{تخریب-محور}{Destruction-oriented}،
\نام{ارزیابی-محور}{Evaluation-oriented} و
\نام{پیشگیری-محور}{Prevention-oriented} تقسیم‌بندی
‌می‌شود\مرجع{laycock1993theory}. برخی فعالیت‌های بارز این دوره‌ها به شرح
زیر است:
\شروع{شمارش}
\فقره{
تا سال ۱۹۵۷، برنامه نویس‌ها پس از نوشتن برنامه، آن را اجرا نموده تا از
کارکرد صحیح آن اطمینان یابند و اگر \واژه{باگ}ی در اجرا پیدا می‌شد، آن را پیدا
کرده و رفع می‌نمودند. این فرآیند ادامه پیدا می‌کرد تا زمانی که احساس کنند
\واژه{باگ}ی باقی نمانده است.
}
\فقره{
در سال ۱۹۵۷، \نام{بیکر}{Charles L. Baker}
برای اولین بار \واژه{تست.ن.ا.} را، به‌عنوان روشی برای
«اثبات عملکرد صحیح» برنامه، از \واژه{دیباگ} تمیز
داد\مرجع{baker1957review}.
}
\فقره{
\نام{دایسترا}{Edsger Dijkstra} در سال ۱۹۶۹ متذکر شد کاربرد تست،
«یافتن \واژه{باگ}ها» است، نه اثبات عملکرد صحیح
برنامه\مرجع{testinghistory}.
در سال ۱۹۸۳، راهنمایی برای \واژه{اعتبار.سنجی}، \واژه{وارسی} و
\واژه{تست.ن.ا.} منتشر شد \مرجع{adrion1982validation} که «تشخیص \واژه{باگ}‌های تحلیل و طراحی» را، علاوه بر تشخیص \واژه{باگ}‌های
پیاده‌سازی، با \واژه{تست.ن.ا.} میسر می‌سازد \مرجع{luo2001software}.
}
\فقره{
در سال ۱۹۸۷، \نام{موسا، ایانینو و اوکمتو}{John D. Musa,
Anthony Iannino, and Kazuhira Okumoto} معیار \واژه{اتکا.پذیری} را
به‌عنوان عاملی کلیدی در «اندازه‌گیری \واژه{کیفیت.ن.ا.}» معرفی
نمودند\مرجع{musa1987software}.
}
\فقره{
در سال ۱۹۹۰، \نام{بیزر}{Boris Beizer} \واژه{تست.ن.ا.} را یکی از
موثرترین عوامل «پیشگیری از \واژه{باگ}»‌ معرفی
می‌کند \مرجع{beizer20672software}.
}
\پایان{شمارش}

پس از رویکردهای مذکور در اواخر دهه ۱۹۹۰، با ظهور \واژه{متدولوژی}‌های جدید مانند \واژه{ایکس.پی.}، که در دسته‌بندی
\واژه{متدولوژی}‌های \واژه{ای.اس.دی.} قرار می‌گیرد، رویکردهای جدیدی نسبت به \واژه{تست.ن.ا.} ایجاد شد.

در \واژه{متدولوژی}‌های سنتی با \واژه{مدل.آبشاری}،
\واژه{تست} یکی از فاز‌های \واژه{اس.دی.پی.} بود که فقط یک بار
انجام می‌شد و پیش‌نیاز آن، اتمام فازهای \واژه{تحلیل}، \واژه{طراحی} و
\واژه{پیاده.سازی} بود. در \واژه{متدولوژی}‌های آبشاری، هر کدام از فازهای
\واژه{تحلیل}، \واژه{طراحی}، \واژه{پیاده.سازی} و \واژه{تست} به‌صورت مجزا
اجرا می‌شد \مرجع{agileVsWaterfall} و \واژه{تست} نقشی موثری در فازهای
تحلیل، طراحی و پیاده‌سازی نداشت.

در \واژه{متدولوژی}‌های چابک، \واژه{تست} به‌صورت
\واژه{تکراری} اجرا می‌شود. در زمان نوشتن \واژه{تست}، توصیف کامل
نرم‌افزار مشخص نیست و در \واژه{تکرار} بعد، ممکن است تغییر کند. پس لازم
است \واژه{تست}‌ها نیز در برابر تغییرات منعطف باشند
\مرجع{ambysoft:agileTesting}. این نیازمندی‌ها نقش \واژه{تست} را در \واژه{اس.دی.پی.} کلیدی‌تر می‌کند. \نام{بک}{Kent Beck} در سال
۲۰۰۲، با معرفی \واژه{تکنیک} \واژه{تی.دی.دی.}، \واژه{تست} را به‌عنوان «محرک توسعه‌ی نرم‌افزار» معرفی
می‌کند و نوشتن \واژه{تست} را پیش‌نیاز 
\واژه{پیاده.سازی} می‌داند. وی «بهبود طراحی نرم‌افزار» را از نتایج بکارگیری این
\واژه{تکنیک} می‌داند \مرجع{beck2003test}. همچنین با بهره‌گیری از این نگرش  در \واژه{متدولوژی}
\واژه{ای.ام.دی.دی.}، از \واژه{تست.ن.ا.} به‌عنوان «\واژه{توصیف}
نرم‌افزار»
بهره گرفته می‌شود \مرجع{agileModeling:specByExample}.

%%% Local Variables:
%%% mode: latex
%%% TeX-master: "../main"
%%% End:

\قسمت{ادبیات مورد نیاز}\برچسب{sec:intro:tdd}

%%% Local Variables:
%%% mode: latex
%%% TeX-master: "../main"
%%% End:

\قسمت{تکنیک \واژه{بی.دی.دی.}}\برچسب{sec:intro:bdd}

%%% Local Variables:
%%% mode: latex
%%% TeX-master: "../main"
%%% End:

\input{introduction/writing}

\input{introduction/structure}

%%% Local Variables:
%%% mode: latex
%%% TeX-master: "../main"
%%% End:
