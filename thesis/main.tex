\documentclass[BScThesis,twoside]{sharifthesis}
%\documentclass[MScThesis,oneside]{sharifthesis}
%\documentclass[PhDThesis,twoside]{sharifthesis}
%\documentclass[PhDThesis,oneside]{sharifthesis}
%\documentclass[PhDProposal,twoside]{sharifthesis}
%\documentclass[PhDProposal,oneside]{sharifthesis}

\def\enSubject{Bachelor of Science Thesis (Information Technology Major), Computer Engineering Department, Sharif University of Technology, Tehran, I. R. Iran}
%\def\enSubject{Master of Science Thesis (Information Technology Major), Computer Engineering Department, Sharif University of Technology, Tehran, I. R. Iran}
%\def\enSubject{Doctor of Philosophy Thesis (Information Technology Major), Computer Engineering Department, Sharif University of Technology, Tehran, I. R. Iran}
%\def\enSubject{Doctor of Philosophy Research Proposal (Information Technology Major), Computer Engineering Department, Sharif University of Technology, Tehran, I. R. Iran}

%do not use newline command (\\) in following definition
\def\enTitle{Title of thesis}
%if title is long and requires to be splitted in two lines, uncomment following two definitions and split title at appropriate location
%\def\enTitleLineOne{First line of the long title}
%\def\enTitleLineTwo{and its continuation on the second line}
\def\enAuthor{Seyed Mehran Kholdi, Mohammad Hossein Sekhavat}
\def\enKeywords{First Key Word, Second Key Word, Final Key Word}


\input{general/preamble}
\addbibresource{resources/resources.bib}


\newcommand{\faKeywords}{تست، رفتار رانه، سناریو، مستقل‌سازی
}
\eqcommand{واژه‌های‌کلیدی}{faKeywords}
\آرم{\درج‌تصویر[scale=.7]{logo}}
\تاریخ{شهریور ۱۳۹۴}
%در دستور زیر، از \\ استفاده نکنید
\عنوان{مستقل‌سازی سناریو از تست رفتار رانه}
%اگر عنوان طولانی بوده (و در عنوان انگلیسی از دو خظ استفاده شده) باید دو خط زیر از کامنت خارج و دو خط عنوان توسط آن‌ها تعریف گردد.
%\عنوانخطیک{خط نخست عنوان طولانی}
%\عنوانخطدو{و ادامه‌ی آن در خط دوم}
\نویسنده{سید مهران خلدی، محمد حسین سخاوت}
\دانشگاه{{\نستعلیق\درشت‌تر دانشگاه صنعتی شریف %
\\[0.6cm]}
دانشکده‌ی مهندسی کامپیوتر}
\دانشگاه‌عادی{دانشگاه صنعتی شریف\\
دانشکده‌ی مهندسی کامپیوتر}
\موضوع{گرایش نرم‌افزار}
\استادراهنما{دکتر سید حسن میریان}
%اگر استاد مشاور ندارید، خط زیر را comment کنید
%همچنین، فراموش نکنید در آخر این پرونده، اطلاعات انگلیسی معادل این دستورها را هم پر کنید
%\استادمشاور{دکتر <نام استاد مشاور>} 

\newcommand{\efootnote}[1]{\footnote{\lr{#1}}}
\newcommand{\ecfootnote}[1]{}


% ================ Correct hyphenations ================
\hyphenation{test}


\makeglossaries
%\includeonly{related_works/related_works}
%\includeonly{evaluation/evaluation}

% ===== DEPRACATED AREA =====
% Following commands are provided to make older documents compilable.
% These commands are depracated and should not be used in new documents.
\newcommand{\ترجمه‌ج}[2]
{\ترجمه[#1‌ها]{#1}{#2}}
\newcommand{\ترجمه‌جمع}[3]
{\ترجمه[#3]{#1}{#2}}
\newcommand{\برگردان}[3]
{\ترجمه{#1}{#3}\زیرنویس{#2}}
\eqcommand{اسم}
{نام}
% ===== END OF DEPRACATED AREA =====

\شروع{نوشتار}


\newcommand{\StartDocument}{\frontmatter \baselineskip1.2\baselineskip \pagestyle{empty} \null \vfill
\شروع{وسط‌چین}
{\نستعلیق‌درشت بسم اللّه الرّحمن الرّحیم}
\پایان{وسط‌چین}
\vfill}

%the initial title is supposed to be printed on the cover.
%for non final version, you can leave following commands as is to create only one title page (printed on paper)
%for final version you need to swap folowing commented/uncommented makethesistitle commands to achieve this order: title on the cover THEN in the name of god page THEN another title page but this time printed on paper
\makethesistitle
\StartDocument
%\makethesistitle

\setlength{\baselineskip}{0.9cm}
\begin{comment}
\فصل*{پیش‌گفتار}
\thispagestyle{pagenumberonlyPagestyle}
پیشگفتار اختیاری است. در صورت تمایل به نگارش پیش‌گفتار، محیط کامنت که آن را دربرگرفته باید حذف شود.
\end{comment}

\شروع{قدردانی}
از جناب آقای دکتر مصطفی مهدیه به خاطر کمک‌ها، مشورت‌ها و پیگیری‌های
عالمانه و دلسوزانه‌ی ایشان قدردانی می‌نماییم.

\پایان{قدردانی}
% END OF COMMENT FOR PhD Proposal.

\شروع{چکیده}{\واژه‌های‌کلیدی}
% abstract ...
% write it at the end...
%persian abstract

در این پایان‌نامه امکان و لزوم مستقل‌سازی سناریوی مورد نیاز برای ارضای
شرط \lr{given}، از توصیف شرط \lr{given}، در تست رفتاررانه بررسی
شده‌است. همچنین
معماری‌ای سه‌لایه برای پیاده‌سازی این معماری پیشنهاد گشته و مستندات پیاده‌سازی تست یک
سیستم نمونه با این معماری، شرح داده‌شده.

در این معماری، لایه‌ی پایینی وظیفه‌ی توصیف تست‌ها را دارد، لایه‌ی میانی
موازی با برنامه‌ی اصلی، وظیفه‌ی اجرای تست‌ها را دارد و لایه‌ی بالایی وظیفه‌ی
کنش سناریوی مناسب در برنامه، به‌طوریکه شرایط تست‌ها محقق شود، را
دارد.

\پایان{چکیده}


\setlength{\baselineskip}{0.9cm}
\pagenumbering{tartibi}\tableofcontents\listoftables\listoffigures
%list of abbreviations may be added here...


\PrepareForMainContent
\input{general/thesis_content}



\PrepareForBibliography

\setlatintextfont[Scale=1]{Linux Libertine}
\setlength{\baselineskip}{0.8cm}
%\setromantextfont[Scale=1.2]{XB Niloofar}

%\bibliographystyle{IEEEtran}
%\bibliographystyle{is-unsrt}
%\bibliographystyle{ieeetr-fa}
%\bibliographystyle{amsplain}

%\bibliography{resources/resources}
\latin
\printbibliography[title=\bibliographytitle,heading=bibintoc]
\persian

% glossaries
{\cleardoublepage\setlength{\baselineskip}{1cm}\printpersianglossary\cleardoublepage\printenglishglossary}


\PrepareForLatinPages
\date{April 2014}
\logo{\includegraphics[scale=.4]{logo-en}}
\title{\sffamily\enTitle}
% uncomment following lines only if you have defined commands for two-lines-title at the beginning of this file
%\titlelineone{\enTitleLineOne}
%\titlelinetwo{\enTitleLineTwo}
\author{\sffamily\enAuthor}
\university{\normalfont\bfseries Sharif University of Technology\\Computer Engineering Department}
\subject{Your Major in English Language}
\supervisor{\sffamily Dr. <name of advisor prof.>}
%If you don't have a consultant professor, comment following line
\consult{\sffamily Dr. <name of consultant prof.>}
\begin{abstract}{\enKeywords}
\input{general/abstract-en}
\end{abstract}
\makethesistitle
\پایان{نوشتار}

%%% Local Variables:
%%% mode: latex
%%% TeX-master: t
%%% End:
